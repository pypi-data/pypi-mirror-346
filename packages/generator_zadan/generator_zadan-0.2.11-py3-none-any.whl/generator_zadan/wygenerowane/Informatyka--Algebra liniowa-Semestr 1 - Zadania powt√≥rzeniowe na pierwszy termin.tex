% !TeX spellcheck = pl_PL-Polish
\documentclass[a4paper,12pt]{article}
\linespread{1.3} %odstepy miedzy liniami
\usepackage[a4paper, lmargin=2cm, rmargin=2cm, tmargin=2cm, bmargin=2cm]{geometry}
\usepackage{amsfonts}
\usepackage{amsmath}
\usepackage{color}
\usepackage{enumitem}
\usepackage{fancyhdr}
\usepackage{float}
\usepackage{graphicx}
\usepackage{ifthen}
\usepackage[utf8]{inputenc}
\usepackage{lmodern}
\usepackage{ocgx}
\usepackage{polski}
\usepackage{tcolorbox}
\tcbuselibrary{most}
\tcbuselibrary{skins}
\tcbuselibrary{raster}
% brak - bez odpowiedzi i bez miejsca, white - bez odpowiedzi z miejscem, red = odpowiedzi ukryte ale dostepne
\newcommand{\kolorodpowiedzi}{blue}
\renewcommand{\footrulewidth}{0.4pt}% linia pozioma na końcu strony - default is 0pt
\DeclareFontShape{OMX}{cmex}{m}{n}
    {<-7.5> cmex7
    <7.5-8.5> cmex8
    <8.5-9.5> cmex9
    <9.5-> cmex10}{}
\DeclareSymbolFont{largesymbols}{OMX}{cmex}{m}{n}


\newcommand{\ukryte}{1}  % domyślnie odpowiedzi są do pokazywania po kliknięciu
\ifthenelse{\equal{\kolorodpowiedzi}{red}}  % ukrywamy od pokazywania gdy kolor jest red
	{\renewcommand{\ukryte}{0}}{}

\newcommand{\zOdpowiedziami}[3]{
	\ifthenelse{\equal{#1}{brak}}{}{
		\ifthenelse{\equal{#1}{white}}{\vphantom{#3}}{
			\switchocg{#2}{\textcolor{\kolorodpowiedzi}{\\Rozwiązanie: }}
				\begin{ocg}{Warstwa odpowiedzi}{#2}{\ukryte}
					\textcolor{\kolorodpowiedzi}{#3}
				\end{ocg}}}}

\begin{document}
    \pagestyle{fancy}
    \setlength{\headheight}{27.29453pt}
    \fancyhead{}
    \fancyhead[L]{\textbf{Informatyka\\Algebra liniowa - }}
    \fancyhead[R]{\textbf{Zadania powtórzeniowe na pierwszy termin\\ 15 grudnia 2024}}
    \fancyfoot{}
    \fancyfoot[R]{\tiny\textbf{15 grudnia 2024, 21:57}}
%    \hspace{1cm}%--------------------------------------------------------------------------------
    \fancyhead[C]{\textbf{Zestaw nr 1}}
    \begin{enumerate}[label= \textbf{Zadanie \arabic*. }, leftmargin=1cm, align=left, itemsep=0pt]
		\item Rozwiązać równanie w zbiorze liczb zespolonych.\ Sprawdzić rozwiązanie.
			\[
				\left(6 + 5i\right)z = -6 - 2i+\left(6 + 6i\right)z
			\]
			\zOdpowiedziami{\kolorodpowiedzi}{ocg0}
				{$z=2 - 6i.$}

		\item Rozwiązać równanie w zbiorze liczb zespolonych.\ Sprawdzić jedno z rozwiązań.
			\[
				\left(1 - i\right)z^2 + \left(1 + 11 i\right) z + \left(-18 - 14 i\right)=0
			\]
			\zOdpowiedziami{\kolorodpowiedzi}{ocg1}
				{$\Delta = 8+6i, \quad \sqrt{\Delta}=\pm( 3 + i), \quad z_{1}=2 - 4 i, \quad z_{2}=3 - 2 i$}

