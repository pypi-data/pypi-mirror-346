% !TeX spellcheck = pl_PL-Polish
\documentclass[a4paper,12pt]{article}
\linespread{1.3} %odstepy miedzy liniami
\usepackage[a4paper, lmargin=2cm, rmargin=2cm, tmargin=2cm, bmargin=2cm]{geometry}
\usepackage{amsfonts}
\usepackage{amsmath}
\usepackage{color}
\usepackage{enumitem}
\usepackage{fancyhdr}
\usepackage{float}
\usepackage{graphicx}
\usepackage{ifthen}
\usepackage[utf8]{inputenc}
\usepackage{lmodern}
\usepackage{ocgx}
\usepackage{polski}
\usepackage{tcolorbox}
\tcbuselibrary{most}
\tcbuselibrary{skins}
\tcbuselibrary{raster}
% brak - bez odpowiedzi i bez miejsca, white - bez odpowiedzi z miejscem, red = odpowiedzi ukryte ale dostepne
\newcommand{\kolorodpowiedzi}{blue}
\renewcommand{\footrulewidth}{0.4pt}% linia pozioma na końcu strony - default is 0pt
\DeclareFontShape{OMX}{cmex}{m}{n}
    {<-7.5> cmex7
    <7.5-8.5> cmex8
    <8.5-9.5> cmex9
    <9.5-> cmex10}{}
\DeclareSymbolFont{largesymbols}{OMX}{cmex}{m}{n}


\newcommand{\ukryte}{1}  % domyślnie odpowiedzi są do pokazywania po kliknięciu
\ifthenelse{\equal{\kolorodpowiedzi}{red}}  % ukrywamy od pokazywania gdy kolor jest red
	{\renewcommand{\ukryte}{0}}{}

\newcommand{\zOdpowiedziami}[3]{
	\ifthenelse{\equal{#1}{brak}}{}{
		\ifthenelse{\equal{#1}{white}}{\vphantom{#3}}{
			\switchocg{#2}{\textcolor{\kolorodpowiedzi}{\\Rozwiązanie: }}
				\begin{ocg}{Warstwa odpowiedzi}{#2}{\ukryte}
					\textcolor{\kolorodpowiedzi}{#3}
				\end{ocg}}}}

\begin{document}
    \pagestyle{fancy}
    \setlength{\headheight}{27.29453pt}
    \fancyhead{}
    \fancyhead[L]{\textbf{Oceanotechnika\\Matematyka 1 - }}
    \fancyhead[R]{\textbf{Zadania powtórzeniowe\\ 11 listopada 2024}}
    \fancyfoot{}
    \fancyfoot[R]{\tiny\textbf{11 listopada 2024, 17:39}}
%    \hspace{1cm}%--------------------------------------------------------------------------------
    \fancyhead[C]{\textbf{Zestaw nr 1}}
    \begin{enumerate}[label= \textbf{Zadanie \arabic*. }, leftmargin=1cm, align=left, itemsep=0pt]
		\item Rozwiązać równanie w zbiorze liczb zespolonych.\ Sprawdzić rozwiązanie.
			\[
				-5 + 9i+\left(8 + 5i\right)z = \left(8 + 4i\right)z
			\]
			\zOdpowiedziami{\kolorodpowiedzi}{ocg0}
				{$z=-9 - 5i.$}

		\item Rozwiązać równanie w zbiorze liczb zespolonych
			\[
				z \left(-6 - 2 i\right) + \left(-1 - 5 i\right) \overline{z} + 3 + 5 i = 0
			\]
			\zOdpowiedziami{\kolorodpowiedzi}{ocg1}
				{$\left(-1 - 5 i\right) \left(x - i y\right) + \left(-6 - 2 i\right) \left(x + i y\right) + 3 + 5 i = 0,$ \\ 
			$\left(-1 - i\right) \left(7 x + y \left(4 + i\right) - 4 - i\right) = 0,$\\
			$\left\{
				\begin{array}{c}
					- 7 x - 3 y + 3 = 0\\
					- 7 x - 5 y + 5 = 0
				\end{array}
			\right.$ \\
			$z = \left\{ x : 0, \  y : 1\right\}.$}

		\item Rozwiązać równanie w zbiorze liczb zespolonych.\ Sprawdzić jedno z rozwiązań.
			\[
				\left(1 - 2 i\right)z^2 + \left(9 - 3 i\right) z + \left(8 + 4 i\right)=0
			\]
			\zOdpowiedziami{\kolorodpowiedzi}{ocg2}
				{$\Delta = 8-6i, \quad \sqrt{\Delta}=\pm( 3 - i), \quad z_{1}=-2 - 2 i, \quad z_{2}=-1 - i$}

		\item Dla jakich rzeczywistych wartości parametru $x$ wyznacznik macierzy $A$ jest różny od zera?
			\[
				\textnormal{A=}
				\left[\begin{matrix}x - 4 & 2 x - 1 & -1\\-2 & x + 1 & -3\\-3 & -3 & -1\end{matrix}\right]
			\]
			\zOdpowiedziami{\kolorodpowiedzi}{ocg3}
				{$\det A=- x^{2} + 5 x + 24 \neq 0, \quad x\neq -3, \ x\neq 8, \ $}

		\item Dla jakich rzeczywistych wartości parametru $x$ macierz $A$ posiada odwrotność?
			\[
				\textnormal{A=}\left[\begin{matrix}x + 2 & 2 & 3\\x + 2 & x + 4 & x + 4\\0 & 3 & 2\end{matrix}\right]
			\]
		Wyznaczyć macierz odwrotną dla $x=2.$ Wykonać sprawdzenie.\\
			\zOdpowiedziami{\kolorodpowiedzi}{ocg4}
				{$\det A=- x^{2} - x + 2\neq 0, \quad 
				x\neq -2\ \textnormal{oraz} \ x\neq 1,$\\
				$A(2)= \left[\begin{matrix}4 & 2 & 3\\4 & 6 & 6\\0 & 3 & 2\end{matrix}\right],\ 
				\det A(2)=-4,\ 
				A^{-1}=- \frac{1}{4}\left[\begin{matrix}-6 & 5 & -6\\-8 & 8 & -12\\12 & -12 & 16\end{matrix}\right].$}

		\item Rozwiązać równanie:
			\[
				\left[\begin{matrix}0 & 0 & 2 & 1 & 1\\1 & 2 & 2 & 1 & -1\end{matrix}\right]\cdot 
				\left[\begin{matrix}-2 & 1 & 0 & 0 & -2\\1 & -2 & 1 & 0 & 0\end{matrix}\right]^T + 
				2X=
				\left[\begin{matrix}-1 & 2\\2 & 0\\1 & 1\\2 & 1\\0 & -1\end{matrix}\right]^T \cdot
				\left[\begin{matrix}1 & 1\\-1 & 0\\-2 & 1\\-1 & -1\\-2 & 0\end{matrix}\right] \cdot X
			\]
			\zOdpowiedziami{\kolorodpowiedzi}{ocg5}
				{$ \left[\begin{matrix}-2 & 2\\2 & -1\end{matrix}\right] + 
				2X=
				\left[\begin{matrix}-7 & -2\\1 & 2\end{matrix}\right] \cdot X, \quad 
				\left[\begin{matrix}-2 & 2\\2 & -1\end{matrix}\right] = 
				\left[\begin{matrix}-9 & -2\\1 & 0\end{matrix}\right] \cdot X $ \\ 
				$X=\frac{1}{2}\left[\begin{matrix}4 & -2\\-16 & 7\end{matrix}\right].$}

		\item Wyznaczyć prostą regresji dla punktów 
			\[
				(2,-1),\  (-4,-3),\  (-3,-5),\  (0,0),\  (3,1).
			\]
			\zOdpowiedziami{\kolorodpowiedzi}{ocg6}
				{	\begin{tabular}{p{0.5\textwidth}p{0.3\textwidth}}
				\vspace{0pt}
				\[
					\left[\begin{matrix}2 & 1\\-4 & 1\\-3 & 1\\0 & 1\\3 & 1\end{matrix}\right]
					\cdot \left[\begin{matrix} a \\ b \end{matrix}\right]
					= \left[\begin{matrix}-1\\-3\\-5\\0\\1\end{matrix}\right]
					\quad \Biggm/ \cdot \left(\left(A^T A \right)^{-1} A^T \right)_L
				\]
				\[
					 \left[\begin{matrix} a \\ b \end{matrix}\right] = 
					 \left[\begin{matrix} \frac{2}{3} \\ - \frac{4}{3} \end{matrix}\right]
				\]
				&
					\raisebox{-4.9cm}{\resizebox{5.2cm}{!}{\includegraphics{../pics/regresja6}}}
				\end{tabular}
			}

    \end{enumerate}
\newpage % \hspace{1cm}
%--------------------------------------------------------------------------------
    \fancyhead[C]{\textbf{Zestaw nr 2}}
    \begin{enumerate}[label= \textbf{Zadanie \arabic*. }, leftmargin=1cm, align=left, itemsep=0pt]
		\item Rozwiązać równanie w zbiorze liczb zespolonych.\ Sprawdzić rozwiązanie.
			\[
				\left(3 - 2i\right)z = -4 + 6i+\left(3 - 3i\right)z
			\]
			\zOdpowiedziami{\kolorodpowiedzi}{ocg7}
				{$z=6 + 4i.$}

		\item Rozwiązać równanie w zbiorze liczb zespolonych
			\[
				z \left(1 - 6 i\right) + \left(3 + 4 i\right) \overline{z} - 2 + 4 i = 0
			\]
			\zOdpowiedziami{\kolorodpowiedzi}{ocg8}
				{$\left(3 + 4 i\right) \left(x - i y\right) + \left(1 - 6 i\right) \left(x + i y\right) - 2 + 4 i = 0,$ \\ 
			$2 \left(x \left(2 - i\right) + y \left(5 - i\right) - 1 + 2 i\right) = 0,$\\
			$\left\{
				\begin{array}{c}
					4 x + 10 y - 2 = 0\\
					- 2 x - 2 y + 4 = 0
				\end{array}
			\right.$ \\
			$z = \left\{ x : 3, \  y : -1\right\}.$}

		\item Rozwiązać równanie w zbiorze liczb zespolonych.\ Sprawdzić jedno z rozwiązań.
			\[
				\left(1 + 3 i\right)z^2 + \left(-7 + 9 i\right) z + \left(-10\right)=0
			\]
			\zOdpowiedziami{\kolorodpowiedzi}{ocg9}
				{$\Delta = 8-6i, \quad \sqrt{\Delta}=\pm( 3 - i), \quad z_{1}=-1 - 2 i, \quad z_{2}=-1 - i$}

		\item Dla jakich rzeczywistych wartości parametru $x$ wyznacznik macierzy $A$ jest różny od zera?
			\[
				\textnormal{A=}
				\left[\begin{matrix}x - 1 & x & x + 4\\1 & x + 3 & -1\\-1 & 4 & -4\end{matrix}\right]
			\]
			\zOdpowiedziami{\kolorodpowiedzi}{ocg10}
				{$\det A=- 3 x^{2} + 12 x + 36 \neq 0, \quad x\neq -2, \ x\neq 6, \ $}

		\item Dla jakich rzeczywistych wartości parametru $x$ macierz $A$ posiada odwrotność?
			\[
				\textnormal{A=}\left[\begin{matrix}x + 2 & 2 & -4\\2 x + 1 & 0 & x - 1\\-3 & 1 & -2\end{matrix}\right]
			\]
		Wyznaczyć macierz odwrotną dla $x=-1.$ Wykonać sprawdzenie.\\
			\zOdpowiedziami{\kolorodpowiedzi}{ocg11}
				{$\det A=- x^{2} - 7 x + 8\neq 0, \quad 
				x\neq -8\ \textnormal{oraz} \ x\neq 1,$\\
				$A(-1)= \left[\begin{matrix}1 & 2 & -4\\-1 & 0 & -2\\-3 & 1 & -2\end{matrix}\right],\ 
				\det A(-1)=14,\ 
				A^{-1}=\frac{1}{14}\left[\begin{matrix}2 & 0 & -4\\4 & -14 & 6\\-1 & -7 & 2\end{matrix}\right].$}

		\item Rozwiązać równanie:
			\[
				\left[\begin{matrix}-1 & 0 & 1 & -1 & 0\\1 & 2 & 2 & -1 & 0\end{matrix}\right]\cdot 
				\left[\begin{matrix}-1 & 0 & 1 & 1 & 1\\1 & -2 & -1 & -2 & 0\end{matrix}\right]^T + 
				2X=
				X \cdot \left[\begin{matrix}0 & -1\\1 & 0\\2 & 1\\0 & 1\\2 & -1\end{matrix}\right]^T \cdot
				\left[\begin{matrix}0 & 0\\0 & 1\\-1 & 1\\-1 & 1\\-1 & 0\end{matrix}\right] 
			\]
			\zOdpowiedziami{\kolorodpowiedzi}{ocg12}
				{$ \left[\begin{matrix}1 & 0\\0 & -3\end{matrix}\right] + 
				2X=
				X \cdot \left[\begin{matrix}-4 & 3\\-1 & 2\end{matrix}\right] , \quad 
				\left[\begin{matrix}1 & 0\\0 & -3\end{matrix}\right] = 
				X \cdot \left[\begin{matrix}-6 & 3\\-1 & 0\end{matrix}\right] $ \\ 
				$X=\frac{1}{3}\left[\begin{matrix}0 & -3\\-3 & 18\end{matrix}\right].$}

		\item Wyznaczyć prostą regresji dla punktów 
			\[
				(-2,-3),\  (4,5),\  (1,4),\  (-3,-5),\  (0,-1).
			\]
			\zOdpowiedziami{\kolorodpowiedzi}{ocg13}
				{	\begin{tabular}{p{0.5\textwidth}p{0.3\textwidth}}
				\vspace{0pt}
				\[
					\left[\begin{matrix}-2 & 1\\4 & 1\\1 & 1\\-3 & 1\\0 & 1\end{matrix}\right]
					\cdot \left[\begin{matrix} a \\ b \end{matrix}\right]
					= \left[\begin{matrix}-3\\5\\4\\-5\\-1\end{matrix}\right]
					\quad \Biggm/ \cdot \left(\left(A^T A \right)^{-1} A^T \right)_L
				\]
				\[
					 \left[\begin{matrix} a \\ b \end{matrix}\right] = 
					 \left[\begin{matrix} \frac{3}{2} \\ 0 \end{matrix}\right]
				\]
				&
					\raisebox{-4.9cm}{\resizebox{5.2cm}{!}{\includegraphics{../pics/regresja13}}}
				\end{tabular}
			}

    \end{enumerate}
\newpage % \hspace{1cm}
%--------------------------------------------------------------------------------
    \fancyhead[C]{\textbf{Zestaw nr 3}}
    \begin{enumerate}[label= \textbf{Zadanie \arabic*. }, leftmargin=1cm, align=left, itemsep=0pt]
		\item Rozwiązać równanie w zbiorze liczb zespolonych.\ Sprawdzić rozwiązanie.
			\[
				\left(6 - 3i\right)z = -6 - 2i+\left(6 - 2i\right)z
			\]
			\zOdpowiedziami{\kolorodpowiedzi}{ocg14}
				{$z=2 - 6i.$}

		\item Rozwiązać równanie w zbiorze liczb zespolonych
			\[
				z \left(-4 - i\right) + \left(2 + 4 i\right) \overline{z} - 4 + 3 i = 0
			\]
			\zOdpowiedziami{\kolorodpowiedzi}{ocg15}
				{$\left(2 + 4 i\right) \left(x - i y\right) + \left(-4 - i\right) \left(x + i y\right) - 4 + 3 i = 0,$ \\ 
			$- x \left(2 - 3 i\right) - y \left(-5 + 6 i\right) - 4 + 3 i = 0,$\\
			$\left\{
				\begin{array}{c}
					- 2 x + 5 y - 4 = 0\\
					3 x - 6 y + 3 = 0
				\end{array}
			\right.$ \\
			$z = \left\{ x : 3, \  y : 2\right\}.$}

		\item Rozwiązać równanie w zbiorze liczb zespolonych.\ Sprawdzić jedno z rozwiązań.
			\[
				\left(1 + 2 i\right)z^2 + \left(-7 + 11 i\right) z + \left(-18 - i\right)=0
			\]
			\zOdpowiedziami{\kolorodpowiedzi}{ocg16}
				{$\Delta = -8-6i, \quad \sqrt{\Delta}=\pm( 1 - 3 i), \quad z_{1}=-2 - 3 i, \quad z_{2}=-1 - 2 i$}

		\item Dla jakich rzeczywistych wartości parametru $x$ wyznacznik macierzy $A$ jest różny od zera?
			\[
				\textnormal{A=}
				\left[\begin{matrix}x + 3 & x + 3\\x - 3 & -2\end{matrix}\right]
			\]
			\zOdpowiedziami{\kolorodpowiedzi}{ocg17}
				{$\det A=- x^{2} - 2 x + 3 \neq 0, \quad x\neq -3, \ x\neq 1, \ $}

		\item Dla jakich rzeczywistych wartości parametru $x$ macierz $A$ posiada odwrotność?
			\[
				\textnormal{A=}\left[\begin{matrix}-1 & 4 & 3\\x + 1 & 0 & 1\\x - 1 & x + 2 & x + 1\end{matrix}\right]
			\]
		Wyznaczyć macierz odwrotną dla $x=2.$ Wykonać sprawdzenie.\\
			\zOdpowiedziami{\kolorodpowiedzi}{ocg18}
				{$\det A=- x^{2} + 6 x\neq 0, \quad 
				x\neq 0\ \textnormal{oraz} \ x\neq 6,$\\
				$A(2)= \left[\begin{matrix}-1 & 4 & 3\\3 & 0 & 1\\1 & 4 & 3\end{matrix}\right],\ 
				\det A(2)=8,\ 
				A^{-1}=\frac{1}{8}\left[\begin{matrix}-4 & 0 & 4\\-8 & -6 & 10\\12 & 8 & -12\end{matrix}\right].$}

		\item Rozwiązać równanie:
			\[
				\left[\begin{matrix}1 & 1 & 2 & 1 & 0\\0 & 0 & 1 & 1 & -1\end{matrix}\right]\cdot 
				\left[\begin{matrix}0 & 1 & -1 & 0 & -2\\0 & -1 & -2 & 1 & -2\end{matrix}\right]^T + 
				2X=
				X \cdot \left[\begin{matrix}1 & -1\\-1 & 1\\-1 & 2\\0 & 1\\-1 & 1\end{matrix}\right]^T \cdot
				\left[\begin{matrix}1 & 0\\-1 & 0\\1 & 0\\0 & -2\\0 & -1\end{matrix}\right] 
			\]
			\zOdpowiedziami{\kolorodpowiedzi}{ocg19}
				{$ \left[\begin{matrix}-1 & -4\\1 & 1\end{matrix}\right] + 
				2X=
				X \cdot \left[\begin{matrix}1 & 1\\0 & -3\end{matrix}\right] , \quad 
				\left[\begin{matrix}-1 & -4\\1 & 1\end{matrix}\right] = 
				X \cdot \left[\begin{matrix}-1 & 1\\0 & -5\end{matrix}\right] $ \\ 
				$X=\frac{1}{5}\left[\begin{matrix}5 & 5\\-5 & -2\end{matrix}\right].$}

		\item Wyznaczyć prostą regresji dla punktów 
			\[
				(-5,-3),\  (-4,-5),\  (-2,-2),\  (1,5),\  (-1,1).
			\]
			\zOdpowiedziami{\kolorodpowiedzi}{ocg20}
				{	\begin{tabular}{p{0.5\textwidth}p{0.3\textwidth}}
				\vspace{0pt}
				\[
					\left[\begin{matrix}-5 & 1\\-4 & 1\\-2 & 1\\1 & 1\\-1 & 1\end{matrix}\right]
					\cdot \left[\begin{matrix} a \\ b \end{matrix}\right]
					= \left[\begin{matrix}-3\\-5\\-2\\5\\1\end{matrix}\right]
					\quad \Biggm/ \cdot \left(\left(A^T A \right)^{-1} A^T \right)_L
				\]
				\[
					 \left[\begin{matrix} a \\ b \end{matrix}\right] = 
					 \left[\begin{matrix} \frac{3}{2} \\ \frac{5}{2} \end{matrix}\right]
				\]
				&
					\raisebox{-4.9cm}{\resizebox{5.2cm}{!}{\includegraphics{../pics/regresja20}}}
				\end{tabular}
			}

    \end{enumerate}
\newpage % \hspace{1cm}
%--------------------------------------------------------------------------------
    \fancyhead[C]{\textbf{Zestaw nr 4}}
    \begin{enumerate}[label= \textbf{Zadanie \arabic*. }, leftmargin=1cm, align=left, itemsep=0pt]
		\item Rozwiązać równanie w zbiorze liczb zespolonych.\ Sprawdzić rozwiązanie.
			\[
				\left(3 - 4i\right)z = 4 - 2i+\left(2 - i\right)z
			\]
			\zOdpowiedziami{\kolorodpowiedzi}{ocg21}
				{$z=1 + i.$}

		\item Rozwiązać równanie w zbiorze liczb zespolonych
			\[
				z \left(3 + 5 i\right) + \left(-4 + 3 i\right) \overline{z} + 5 + 5 i = 0
			\]
			\zOdpowiedziami{\kolorodpowiedzi}{ocg22}
				{$\left(-4 + 3 i\right) \left(x - i y\right) + \left(3 + 5 i\right) \left(x + i y\right) + 5 + 5 i = 0,$ \\ 
			$- x \left(1 - 8 i\right) - y \left(2 - 7 i\right) + 5 + 5 i = 0,$\\
			$\left\{
				\begin{array}{c}
					- x - 2 y + 5 = 0\\
					8 x + 7 y + 5 = 0
				\end{array}
			\right.$ \\
			$z = \left\{ x : -5, \  y : 5\right\}.$}

		\item Rozwiązać równanie w zbiorze liczb zespolonych.\ Sprawdzić jedno z rozwiązań.
			\[
				\left(1 - i\right)z^2 + \left(-3 + 9 i\right) z + \left(-4 - 16 i\right)=0
			\]
			\zOdpowiedziami{\kolorodpowiedzi}{ocg23}
				{$\Delta = 8-6i, \quad \sqrt{\Delta}=\pm( 3 - i), \quad z_{1}=2 - 2 i, \quad z_{2}=4 - i$}

		\item Dla jakich rzeczywistych wartości parametru $x$ wyznacznik macierzy $A$ jest różny od zera?
			\[
				\textnormal{A=}
				\left[\begin{matrix}-1 & 2 & 2\\x - 4 & 4 & 4\\2 x + 1 & x + 1 & -1\end{matrix}\right]
			\]
			\zOdpowiedziami{\kolorodpowiedzi}{ocg24}
				{$\det A=2 x^{2} - 8 \neq 0, \quad x\neq -2, \ x\neq 2, \ $}

		\item Dla jakich rzeczywistych wartości parametru $x$ macierz $A$ posiada odwrotność?
			\[
				\textnormal{A=}\left[\begin{matrix}x + 3 & -1\\x + 3 & 2 x - 1\end{matrix}\right]
			\]
		Wyznaczyć macierz odwrotną dla $x=-2.$ Wykonać sprawdzenie.\\
			\zOdpowiedziami{\kolorodpowiedzi}{ocg25}
				{$\det A=2 x^{2} + 6 x\neq 0, \quad 
				x\neq -3\ \textnormal{oraz} \ x\neq 0,$\\
				$A(-2)= \left[\begin{matrix}1 & -1\\1 & -5\end{matrix}\right],\ 
				\det A(-2)=-4,\ 
				A^{-1}=- \frac{1}{4}\left[\begin{matrix}-5 & 1\\-1 & 1\end{matrix}\right].$}

		\item Rozwiązać równanie:
			\[
				\left[\begin{matrix}-1 & 2 & 2 & 1 & 0\\-1 & 2 & 1 & 1 & 2\end{matrix}\right]\cdot 
				\left[\begin{matrix}0 & -1 & 1 & 1 & 0\\1 & 0 & 0 & -1 & 0\end{matrix}\right]^T + 
				3X=
				\left[\begin{matrix}0 & -1\\2 & 2\\2 & 2\\2 & 0\\-1 & 1\end{matrix}\right]^T \cdot
				\left[\begin{matrix}-1 & 0\\-2 & -1\\1 & 0\\0 & -1\\-2 & 0\end{matrix}\right] \cdot X
			\]
			\zOdpowiedziami{\kolorodpowiedzi}{ocg26}
				{$ \left[\begin{matrix}1 & -2\\0 & -2\end{matrix}\right] + 
				3X=
				\left[\begin{matrix}0 & -4\\-3 & -2\end{matrix}\right] \cdot X, \quad 
				\left[\begin{matrix}1 & -2\\0 & -2\end{matrix}\right] = 
				\left[\begin{matrix}-3 & -4\\-3 & -5\end{matrix}\right] \cdot X $ \\ 
				$X=\frac{1}{3}\left[\begin{matrix}-5 & 2\\3 & 0\end{matrix}\right].$}

		\item Wyznaczyć prostą regresji dla punktów 
			\[
				(-2,-1),\  (0,0),\  (-4,-3),\  (-3,-4),\  (5,1).
			\]
			\zOdpowiedziami{\kolorodpowiedzi}{ocg27}
				{	\begin{tabular}{p{0.5\textwidth}p{0.3\textwidth}}
				\vspace{0pt}
				\[
					\left[\begin{matrix}-2 & 1\\0 & 1\\-4 & 1\\-3 & 1\\5 & 1\end{matrix}\right]
					\cdot \left[\begin{matrix} a \\ b \end{matrix}\right]
					= \left[\begin{matrix}-1\\0\\-3\\-4\\1\end{matrix}\right]
					\quad \Biggm/ \cdot \left(\left(A^T A \right)^{-1} A^T \right)_L
				\]
				\[
					 \left[\begin{matrix} a \\ b \end{matrix}\right] = 
					 \left[\begin{matrix} \frac{1}{2} \\ -1 \end{matrix}\right]
				\]
				&
					\raisebox{-4.9cm}{\resizebox{5.2cm}{!}{\includegraphics{../pics/regresja27}}}
				\end{tabular}
			}

    \end{enumerate}
\newpage % \hspace{1cm}
%--------------------------------------------------------------------------------
    \fancyhead[C]{\textbf{Zestaw nr 5}}
    \begin{enumerate}[label= \textbf{Zadanie \arabic*. }, leftmargin=1cm, align=left, itemsep=0pt]
		\item Rozwiązać równanie w zbiorze liczb zespolonych.\ Sprawdzić rozwiązanie.
			\[
				-4 - 4i+\left(7 - 5i\right)z = \left(7 - 3i\right)z
			\]
			\zOdpowiedziami{\kolorodpowiedzi}{ocg28}
				{$z=-2 + 2i.$}

		\item Rozwiązać równanie w zbiorze liczb zespolonych
			\[
				z \left(-4 + 2 i\right) + \left(-1 - 3 i\right) \overline{z} + 5 + 7 i = 0
			\]
			\zOdpowiedziami{\kolorodpowiedzi}{ocg29}
				{$\left(-1 - 3 i\right) \left(x - i y\right) + \left(-4 + 2 i\right) \left(x + i y\right) + 5 + 7 i = 0,$ \\ 
			$\left(-1 - i\right) \left(x \left(3 - 2 i\right) + y \left(4 - i\right) - 6 - i\right) = 0,$\\
			$\left\{
				\begin{array}{c}
					- 5 x - 5 y + 5 = 0\\
					- x - 3 y + 7 = 0
				\end{array}
			\right.$ \\
			$z = \left\{ x : -2, \  y : 3\right\}.$}

		\item Rozwiązać równanie w zbiorze liczb zespolonych.\ Sprawdzić jedno z rozwiązań.
			\[
				\left(1 - i\right)z^2 + \left(-5 + 13 i\right) z + \left(-34 i\right)=0
			\]
			\zOdpowiedziami{\kolorodpowiedzi}{ocg30}
				{$\Delta = -8+6i, \quad \sqrt{\Delta}=\pm( 1 + 3 i), \quad z_{1}=4 - i, \quad z_{2}=5 - 3 i$}

		\item Dla jakich rzeczywistych wartości parametru $x$ wyznacznik macierzy $A$ jest różny od zera?
			\[
				\textnormal{A=}
				\left[\begin{matrix}x - 1 & -1 & x - 4\\2 x + 1 & -3 & -1\\3 & -1 & 3\end{matrix}\right]
			\]
			\zOdpowiedziami{\kolorodpowiedzi}{ocg31}
				{$\det A=- 2 x^{2} + 12 x - 16 \neq 0, \quad x\neq 2, \ x\neq 4, \ $}

		\item Dla jakich rzeczywistych wartości parametru $x$ macierz $A$ posiada odwrotność?
			\[
				\textnormal{A=}\left[\begin{matrix}x - 1 & x + 3\\1 & 2 x - 3\end{matrix}\right]
			\]
		Wyznaczyć macierz odwrotną dla $x=2.$ Wykonać sprawdzenie.\\
			\zOdpowiedziami{\kolorodpowiedzi}{ocg32}
				{$\det A=2 x^{2} - 6 x\neq 0, \quad 
				x\neq 0\ \textnormal{oraz} \ x\neq 3,$\\
				$A(2)= \left[\begin{matrix}1 & 5\\1 & 1\end{matrix}\right],\ 
				\det A(2)=-4,\ 
				A^{-1}=- \frac{1}{4}\left[\begin{matrix}1 & -5\\-1 & 1\end{matrix}\right].$}

		\item Rozwiązać równanie:
			\[
				\left[\begin{matrix}-1 & 1 & 1 & -1 & 2\\0 & 0 & 0 & -1 & -1\end{matrix}\right]\cdot 
				\left[\begin{matrix}-1 & -1 & -1 & 1 & -1\\1 & -1 & 0 & 1 & 0\end{matrix}\right]^T + 
				3X=
				\left[\begin{matrix}-1 & -1\\0 & 1\\-1 & 0\\0 & -1\\-1 & 0\end{matrix}\right]^T \cdot
				\left[\begin{matrix}0 & -1\\1 & 1\\-2 & 1\\-1 & -2\\-1 & -1\end{matrix}\right] \cdot X
			\]
			\zOdpowiedziami{\kolorodpowiedzi}{ocg33}
				{$ \left[\begin{matrix}-4 & -3\\0 & -1\end{matrix}\right] + 
				3X=
				\left[\begin{matrix}3 & 1\\2 & 4\end{matrix}\right] \cdot X, \quad 
				\left[\begin{matrix}-4 & -3\\0 & -1\end{matrix}\right] = 
				\left[\begin{matrix}0 & 1\\2 & 1\end{matrix}\right] \cdot X $ \\ 
				$X=\frac{1}{-2}\left[\begin{matrix}-4 & -2\\8 & 6\end{matrix}\right].$}

		\item Wyznaczyć prostą regresji dla punktów 
			\[
				(0,1),\  (1,0),\  (4,-3),\  (-4,5),\  (5,-4).
			\]
			\zOdpowiedziami{\kolorodpowiedzi}{ocg34}
				{	\begin{tabular}{p{0.5\textwidth}p{0.3\textwidth}}
				\vspace{0pt}
				\[
					\left[\begin{matrix}0 & 1\\1 & 1\\4 & 1\\-4 & 1\\5 & 1\end{matrix}\right]
					\cdot \left[\begin{matrix} a \\ b \end{matrix}\right]
					= \left[\begin{matrix}1\\0\\-3\\5\\-4\end{matrix}\right]
					\quad \Biggm/ \cdot \left(\left(A^T A \right)^{-1} A^T \right)_L
				\]
				\[
					 \left[\begin{matrix} a \\ b \end{matrix}\right] = 
					 \left[\begin{matrix} -1 \\ 1 \end{matrix}\right]
				\]
				&
					\raisebox{-4.9cm}{\resizebox{5.2cm}{!}{\includegraphics{../pics/regresja34}}}
				\end{tabular}
			}

    \end{enumerate}
\newpage % \hspace{1cm}
%--------------------------------------------------------------------------------
    \fancyhead[C]{\textbf{Zestaw nr 6}}
    \begin{enumerate}[label= \textbf{Zadanie \arabic*. }, leftmargin=1cm, align=left, itemsep=0pt]
		\item Rozwiązać równanie w zbiorze liczb zespolonych.\ Sprawdzić rozwiązanie.
			\[
				6 + 2i+\left(9 + 8i\right)z = \left(7 + 9i\right)z
			\]
			\zOdpowiedziami{\kolorodpowiedzi}{ocg35}
				{$z=-2 - 2i.$}

		\item Rozwiązać równanie w zbiorze liczb zespolonych
			\[
				z \left(5 + 2 i\right) + \left(-5 - i\right) \overline{z} + 3 - 4 i = 0
			\]
			\zOdpowiedziami{\kolorodpowiedzi}{ocg36}
				{$\left(-5 - i\right) \left(x - i y\right) + \left(5 + 2 i\right) \left(x + i y\right) + 3 - 4 i = 0,$ \\ 
			$i \left(x + y \left(10 + 3 i\right) - 4 - 3 i\right) = 0,$\\
			$\left\{
				\begin{array}{c}
					3 - 3 y = 0\\
					x + 10 y - 4 = 0
				\end{array}
			\right.$ \\
			$z = \left\{ x : -6, \  y : 1\right\}.$}

		\item Rozwiązać równanie w zbiorze liczb zespolonych.\ Sprawdzić jedno z rozwiązań.
			\[
				\left(1 - 2 i\right)z^2 + \left(-8 - 9 i\right) z + \left(-15 + 5 i\right)=0
			\]
			\zOdpowiedziami{\kolorodpowiedzi}{ocg37}
				{$\Delta = 3+4i, \quad \sqrt{\Delta}=\pm( 2 + i), \quad z_{1}=-1 + 2 i, \quad z_{2}=-1 + 3 i$}

		\item Dla jakich rzeczywistych wartości parametru $x$ wyznacznik macierzy $A$ jest różny od zera?
			\[
				\textnormal{A=}
				\left[\begin{matrix}x - 3 & x + 1 & -4\\x + 4 & 0 & x + 4\\3 & 4 & 1\end{matrix}\right]
			\]
			\zOdpowiedziami{\kolorodpowiedzi}{ocg38}
				{$\det A=- 2 x^{2} - 10 x - 8 \neq 0, \quad x\neq -4, \ x\neq -1, \ $}

		\item Dla jakich rzeczywistych wartości parametru $x$ macierz $A$ posiada odwrotność?
			\[
				\textnormal{A=}\left[\begin{matrix}-3 & 2 x + 2 & 1\\-3 & 1 & x + 3\\0 & x + 1 & -2\end{matrix}\right]
			\]
		Wyznaczyć macierz odwrotną dla $x=3.$ Wykonać sprawdzenie.\\
			\zOdpowiedziami{\kolorodpowiedzi}{ocg39}
				{$\det A=3 x^{2} - 3 x\neq 0, \quad 
				x\neq 0\ \textnormal{oraz} \ x\neq 1,$\\
				$A(3)= \left[\begin{matrix}-3 & 8 & 1\\-3 & 1 & 6\\0 & 4 & -2\end{matrix}\right],\ 
				\det A(3)=18,\ 
				A^{-1}=\frac{1}{18}\left[\begin{matrix}-26 & 20 & 47\\-6 & 6 & 15\\-12 & 12 & 21\end{matrix}\right].$}

		\item Rozwiązać równanie:
			\[
				\left[\begin{matrix}0 & 2 & 2 & 0 & 2\\2 & 1 & 1 & -1 & -1\end{matrix}\right]\cdot 
				\left[\begin{matrix}-2 & 0 & -2 & -1 & -1\\0 & -2 & -2 & 1 & 0\end{matrix}\right]^T + 
				3X=
				\left[\begin{matrix}2 & -1\\-1 & 0\\2 & -1\\2 & 1\\2 & -1\end{matrix}\right]^T \cdot
				\left[\begin{matrix}-2 & -2\\-2 & -2\\1 & 1\\-1 & 1\\1 & 0\end{matrix}\right] \cdot X
			\]
			\zOdpowiedziami{\kolorodpowiedzi}{ocg40}
				{$ \left[\begin{matrix}-6 & -8\\-4 & -5\end{matrix}\right] + 
				3X=
				\left[\begin{matrix}0 & 2\\-1 & 2\end{matrix}\right] \cdot X, \quad 
				\left[\begin{matrix}-6 & -8\\-4 & -5\end{matrix}\right] = 
				\left[\begin{matrix}-3 & 2\\-1 & -1\end{matrix}\right] \cdot X $ \\ 
				$X=\frac{1}{5}\left[\begin{matrix}14 & 18\\6 & 7\end{matrix}\right].$}

		\item Wyznaczyć prostą regresji dla punktów 
			\[
				(-1,5),\  (5,-2),\  (-3,4),\  (2,1),\  (0,2).
			\]
			\zOdpowiedziami{\kolorodpowiedzi}{ocg41}
				{	\begin{tabular}{p{0.5\textwidth}p{0.3\textwidth}}
				\vspace{0pt}
				\[
					\left[\begin{matrix}-1 & 1\\5 & 1\\-3 & 1\\2 & 1\\0 & 1\end{matrix}\right]
					\cdot \left[\begin{matrix} a \\ b \end{matrix}\right]
					= \left[\begin{matrix}5\\-2\\4\\1\\2\end{matrix}\right]
					\quad \Biggm/ \cdot \left(\left(A^T A \right)^{-1} A^T \right)_L
				\]
				\[
					 \left[\begin{matrix} a \\ b \end{matrix}\right] = 
					 \left[\begin{matrix} - \frac{5}{6} \\ \frac{5}{2} \end{matrix}\right]
				\]
				&
					\raisebox{-4.9cm}{\resizebox{5.2cm}{!}{\includegraphics{../pics/regresja41}}}
				\end{tabular}
			}

    \end{enumerate}
\newpage % \hspace{1cm}
%--------------------------------------------------------------------------------
    \fancyhead[C]{\textbf{Zestaw nr 7}}
    \begin{enumerate}[label= \textbf{Zadanie \arabic*. }, leftmargin=1cm, align=left, itemsep=0pt]
		\item Rozwiązać równanie w zbiorze liczb zespolonych.\ Sprawdzić rozwiązanie.
			\[
				-4 + 7i+\left(6 - 4i\right)z = \left(4 - 3i\right)z
			\]
			\zOdpowiedziami{\kolorodpowiedzi}{ocg42}
				{$z=3 - 2i.$}

		\item Rozwiązać równanie w zbiorze liczb zespolonych
			\[
				z \left(-6 + 3 i\right) + \left(5 + 6 i\right) \overline{z} + 7 + i = 0
			\]
			\zOdpowiedziami{\kolorodpowiedzi}{ocg43}
				{$\left(5 + 6 i\right) \left(x - i y\right) + \left(-6 + 3 i\right) \left(x + i y\right) + 7 + i = 0,$ \\ 
			$\left(-1 + i\right) \left(x \left(5 - 4 i\right) + y \left(-7 + 4 i\right) - 3 - 4 i\right) = 0,$\\
			$\left\{
				\begin{array}{c}
					- x + 3 y + 7 = 0\\
					9 x - 11 y + 1 = 0
				\end{array}
			\right.$ \\
			$z = \left\{ x : -5, \  y : -4\right\}.$}

		\item Rozwiązać równanie w zbiorze liczb zespolonych.\ Sprawdzić jedno z rozwiązań.
			\[
				\left(1 - i\right)z^2 + \left(-11 - i\right) z + \left(14 + 18 i\right)=0
			\]
			\zOdpowiedziami{\kolorodpowiedzi}{ocg44}
				{$\Delta = -8+6i, \quad \sqrt{\Delta}=\pm( 1 + 3 i), \quad z_{1}=2 + 4 i, \quad z_{2}=3 + 2 i$}

		\item Dla jakich rzeczywistych wartości parametru $x$ wyznacznik macierzy $A$ jest różny od zera?
			\[
				\textnormal{A=}
				\left[\begin{matrix}x + 3 & x + 1 & 4\\x + 4 & -3 & 4\\-2 & x + 4 & -2\end{matrix}\right]
			\]
			\zOdpowiedziami{\kolorodpowiedzi}{ocg45}
				{$\det A=2 x^{2} + 12 x + 10 \neq 0, \quad x\neq -5, \ x\neq -1, \ $}

		\item Dla jakich rzeczywistych wartości parametru $x$ macierz $A$ posiada odwrotność?
			\[
				\textnormal{A=}\left[\begin{matrix}2 x - 2 & x + 1\\2 & x - 1\end{matrix}\right]
			\]
		Wyznaczyć macierz odwrotną dla $x=-1.$ Wykonać sprawdzenie.\\
			\zOdpowiedziami{\kolorodpowiedzi}{ocg46}
				{$\det A=2 x^{2} - 6 x\neq 0, \quad 
				x\neq 0\ \textnormal{oraz} \ x\neq 3,$\\
				$A(-1)= \left[\begin{matrix}-4 & 0\\2 & -2\end{matrix}\right],\ 
				\det A(-1)=8,\ 
				A^{-1}=\frac{1}{8}\left[\begin{matrix}-2 & 0\\-2 & -4\end{matrix}\right].$}

		\item Rozwiązać równanie:
			\[
				\left[\begin{matrix}2 & 2 & 0 & 0 & -1\\-1 & 0 & -1 & 2 & -1\end{matrix}\right]\cdot 
				\left[\begin{matrix}-1 & -1 & -1 & -2 & -2\\1 & -2 & -1 & -1 & 0\end{matrix}\right]^T + 
				3X=
				X \cdot \left[\begin{matrix}2 & 2\\0 & 0\\-1 & 1\\0 & 1\\-1 & 2\end{matrix}\right]^T \cdot
				\left[\begin{matrix}0 & -2\\-2 & -1\\-2 & 1\\0 & -1\\0 & -2\end{matrix}\right] 
			\]
			\zOdpowiedziami{\kolorodpowiedzi}{ocg47}
				{$ \left[\begin{matrix}-2 & -2\\0 & -2\end{matrix}\right] + 
				3X=
				X \cdot \left[\begin{matrix}2 & -3\\-2 & -8\end{matrix}\right] , \quad 
				\left[\begin{matrix}-2 & -2\\0 & -2\end{matrix}\right] = 
				X \cdot \left[\begin{matrix}-1 & -3\\-2 & -11\end{matrix}\right] $ \\ 
				$X=\frac{1}{5}\left[\begin{matrix}18 & -4\\-4 & 2\end{matrix}\right].$}

		\item Wyznaczyć prostą regresji dla punktów 
			\[
				(2,4),\  (1,0),\  (-2,-2),\  (3,2),\  (-4,-4).
			\]
			\zOdpowiedziami{\kolorodpowiedzi}{ocg48}
				{	\begin{tabular}{p{0.5\textwidth}p{0.3\textwidth}}
				\vspace{0pt}
				\[
					\left[\begin{matrix}2 & 1\\1 & 1\\-2 & 1\\3 & 1\\-4 & 1\end{matrix}\right]
					\cdot \left[\begin{matrix} a \\ b \end{matrix}\right]
					= \left[\begin{matrix}4\\0\\-2\\2\\-4\end{matrix}\right]
					\quad \Biggm/ \cdot \left(\left(A^T A \right)^{-1} A^T \right)_L
				\]
				\[
					 \left[\begin{matrix} a \\ b \end{matrix}\right] = 
					 \left[\begin{matrix} 1 \\ 0 \end{matrix}\right]
				\]
				&
					\raisebox{-4.9cm}{\resizebox{5.2cm}{!}{\includegraphics{../pics/regresja48}}}
				\end{tabular}
			}

    \end{enumerate}
\newpage % \hspace{1cm}
%--------------------------------------------------------------------------------
    \fancyhead[C]{\textbf{Zestaw nr 8}}
    \begin{enumerate}[label= \textbf{Zadanie \arabic*. }, leftmargin=1cm, align=left, itemsep=0pt]
		\item Rozwiązać równanie w zbiorze liczb zespolonych.\ Sprawdzić rozwiązanie.
			\[
				6 + 8i+\left(5 + 2i\right)z = \left(7 - 2i\right)z
			\]
			\zOdpowiedziami{\kolorodpowiedzi}{ocg49}
				{$z=-1 + 2i.$}

		\item Rozwiązać równanie w zbiorze liczb zespolonych
			\[
				z \left(-4 - 6 i\right) + \left(6 + 5 i\right) \overline{z} + 7 + i = 0
			\]
			\zOdpowiedziami{\kolorodpowiedzi}{ocg50}
				{$\left(6 + 5 i\right) \left(x - i y\right) + \left(-4 - 6 i\right) \left(x + i y\right) + 7 + i = 0,$ \\ 
			$x \left(2 - i\right) + y \left(11 - 10 i\right) + 7 + i = 0,$\\
			$\left\{
				\begin{array}{c}
					2 x + 11 y + 7 = 0\\
					- x - 10 y + 1 = 0
				\end{array}
			\right.$ \\
			$z = \left\{ x : -9, \  y : 1\right\}.$}

		\item Rozwiązać równanie w zbiorze liczb zespolonych.\ Sprawdzić jedno z rozwiązań.
			\[
				\left(1 + 2 i\right)z^2 + \left(-13 - 6 i\right) z + \left(22 - 6 i\right)=0
			\]
			\zOdpowiedziami{\kolorodpowiedzi}{ocg51}
				{$\Delta = -3+4i, \quad \sqrt{\Delta}=\pm( 1 + 2 i), \quad z_{1}=2 - 2 i, \quad z_{2}=3 - 2 i$}

		\item Dla jakich rzeczywistych wartości parametru $x$ wyznacznik macierzy $A$ jest różny od zera?
			\[
				\textnormal{A=}
				\left[\begin{matrix}-2 & x - 1 & -3\\0 & x - 1 & -1\\x + 1 & x - 1 & 1\end{matrix}\right]
			\]
			\zOdpowiedziami{\kolorodpowiedzi}{ocg52}
				{$\det A=2 x^{2} - 4 x + 2 \neq 0, \quad x\neq 1, \ $}

		\item Dla jakich rzeczywistych wartości parametru $x$ macierz $A$ posiada odwrotność?
			\[
				\textnormal{A=}\left[\begin{matrix}-1 & x - 4\\x + 2 & 2 x + 4\end{matrix}\right]
			\]
		Wyznaczyć macierz odwrotną dla $x=-1.$ Wykonać sprawdzenie.\\
			\zOdpowiedziami{\kolorodpowiedzi}{ocg53}
				{$\det A=4 - x^{2}\neq 0, \quad 
				x\neq -2\ \textnormal{oraz} \ x\neq 2,$\\
				$A(-1)= \left[\begin{matrix}-1 & -5\\1 & 2\end{matrix}\right],\ 
				\det A(-1)=3,\ 
				A^{-1}=\frac{1}{3}\left[\begin{matrix}2 & 5\\-1 & -1\end{matrix}\right].$}

		\item Rozwiązać równanie:
			\[
				\left[\begin{matrix}1 & 2 & 2 & -1 & 0\\-1 & 0 & 1 & 0 & -1\end{matrix}\right]\cdot 
				\left[\begin{matrix}-1 & 0 & 1 & 0 & 0\\0 & 0 & 1 & 0 & -1\end{matrix}\right]^T + 
				3X=
				X \cdot \left[\begin{matrix}0 & 0\\1 & 1\\-1 & -1\\1 & 1\\2 & 0\end{matrix}\right]^T \cdot
				\left[\begin{matrix}-1 & -2\\1 & 0\\1 & -1\\1 & 0\\-1 & 1\end{matrix}\right] 
			\]
			\zOdpowiedziami{\kolorodpowiedzi}{ocg54}
				{$ \left[\begin{matrix}1 & 2\\2 & 2\end{matrix}\right] + 
				3X=
				X \cdot \left[\begin{matrix}-1 & 3\\1 & 1\end{matrix}\right] , \quad 
				\left[\begin{matrix}1 & 2\\2 & 2\end{matrix}\right] = 
				X \cdot \left[\begin{matrix}-4 & 3\\1 & -2\end{matrix}\right] $ \\ 
				$X=\frac{1}{5}\left[\begin{matrix}-4 & -11\\-6 & -14\end{matrix}\right].$}

		\item Wyznaczyć prostą regresji dla punktów 
			\[
				(-1,1),\  (1,-1),\  (-4,4),\  (2,-2),\  (5,-5).
			\]
			\zOdpowiedziami{\kolorodpowiedzi}{ocg55}
				{	\begin{tabular}{p{0.5\textwidth}p{0.3\textwidth}}
				\vspace{0pt}
				\[
					\left[\begin{matrix}-1 & 1\\1 & 1\\-4 & 1\\2 & 1\\5 & 1\end{matrix}\right]
					\cdot \left[\begin{matrix} a \\ b \end{matrix}\right]
					= \left[\begin{matrix}1\\-1\\4\\-2\\-5\end{matrix}\right]
					\quad \Biggm/ \cdot \left(\left(A^T A \right)^{-1} A^T \right)_L
				\]
				\[
					 \left[\begin{matrix} a \\ b \end{matrix}\right] = 
					 \left[\begin{matrix} -1 \\ 0 \end{matrix}\right]
				\]
				&
					\raisebox{-4.9cm}{\resizebox{5.2cm}{!}{\includegraphics{../pics/regresja55}}}
				\end{tabular}
			}

    \end{enumerate}
\newpage % \hspace{1cm}
%--------------------------------------------------------------------------------
    \fancyhead[C]{\textbf{Zestaw nr 9}}
    \begin{enumerate}[label= \textbf{Zadanie \arabic*. }, leftmargin=1cm, align=left, itemsep=0pt]
		\item Rozwiązać równanie w zbiorze liczb zespolonych.\ Sprawdzić rozwiązanie.
			\[
				-4 + 4i+\left(3 + 2i\right)z = \left(3 + 3i\right)z
			\]
			\zOdpowiedziami{\kolorodpowiedzi}{ocg56}
				{$z=4 + 4i.$}

		\item Rozwiązać równanie w zbiorze liczb zespolonych
			\[
				z \left(-5 - 6 i\right) + \left(5 - 5 i\right) \overline{z} - 5 - 5 i = 0
			\]
			\zOdpowiedziami{\kolorodpowiedzi}{ocg57}
				{$\left(5 - 5 i\right) \left(x - i y\right) + \left(-5 - 6 i\right) \left(x + i y\right) - 5 - 5 i = 0,$ \\ 
			$- i \left(11 x + y \left(10 + i\right) + 5 - 5 i\right) = 0,$\\
			$\left\{
				\begin{array}{c}
					y - 5 = 0\\
					- 11 x - 10 y - 5 = 0
				\end{array}
			\right.$ \\
			$z = \left\{ x : -5, \  y : 5\right\}.$}

		\item Rozwiązać równanie w zbiorze liczb zespolonych.\ Sprawdzić jedno z rozwiązań.
			\[
				\left(1 - i\right)z^2 + \left(11 + 5 i\right) z + \left(-2 + 24 i\right)=0
			\]
			\zOdpowiedziami{\kolorodpowiedzi}{ocg58}
				{$\Delta = 8+6i, \quad \sqrt{\Delta}=\pm( 3 + i), \quad z_{1}=-2 - 5 i, \quad z_{2}=-1 - 3 i$}

		\item Dla jakich rzeczywistych wartości parametru $x$ wyznacznik macierzy $A$ jest różny od zera?
			\[
				\textnormal{A=}
				\left[\begin{matrix}3 & x - 1 & -3\\1 & 2 & -1\\-4 & -1 & 3 x - 2\end{matrix}\right]
			\]
			\zOdpowiedziami{\kolorodpowiedzi}{ocg59}
				{$\det A=- 3 x^{2} + 27 x - 42 \neq 0, \quad x\neq 2, \ x\neq 7, \ $}

		\item Dla jakich rzeczywistych wartości parametru $x$ macierz $A$ posiada odwrotność?
			\[
				\textnormal{A=}\left[\begin{matrix}3 & -1 & x - 4\\4 & -4 & 2 x - 4\\x + 2 & 3 & 3\end{matrix}\right]
			\]
		Wyznaczyć macierz odwrotną dla $x=-2.$ Wykonać sprawdzenie.\\
			\zOdpowiedziami{\kolorodpowiedzi}{ocg60}
				{$\det A=2 x^{2} - 14 x - 60\neq 0, \quad 
				x\neq -3\ \textnormal{oraz} \ x\neq 10,$\\
				$A(-2)= \left[\begin{matrix}3 & -1 & -6\\4 & -4 & -8\\0 & 3 & 3\end{matrix}\right],\ 
				\det A(-2)=-24,\ 
				A^{-1}=- \frac{1}{24}\left[\begin{matrix}12 & -15 & -16\\-12 & 9 & 0\\12 & -9 & -8\end{matrix}\right].$}

		\item Rozwiązać równanie:
			\[
				\left[\begin{matrix}2 & 0 & -1 & -1 & 0\\0 & -1 & 1 & -1 & -1\end{matrix}\right]\cdot 
				\left[\begin{matrix}-1 & -2 & 0 & -2 & -1\\-1 & -2 & -2 & 1 & 0\end{matrix}\right]^T + 
				2X=
				X \cdot \left[\begin{matrix}1 & -1\\1 & -1\\-1 & 0\\0 & -1\\-1 & 2\end{matrix}\right]^T \cdot
				\left[\begin{matrix}0 & -2\\-1 & 0\\-1 & -1\\-2 & -1\\1 & -1\end{matrix}\right] 
			\]
			\zOdpowiedziami{\kolorodpowiedzi}{ocg61}
				{$ \left[\begin{matrix}0 & -1\\5 & -1\end{matrix}\right] + 
				2X=
				X \cdot \left[\begin{matrix}-1 & 0\\5 & 1\end{matrix}\right] , \quad 
				\left[\begin{matrix}0 & -1\\5 & -1\end{matrix}\right] = 
				X \cdot \left[\begin{matrix}-3 & 0\\5 & -1\end{matrix}\right] $ \\ 
				$X=\frac{1}{3}\left[\begin{matrix}5 & 3\\0 & 3\end{matrix}\right].$}

		\item Wyznaczyć prostą regresji dla punktów 
			\[
				(4,-3),\  (-3,2),\  (-4,5),\  (1,-1),\  (3,-4).
			\]
			\zOdpowiedziami{\kolorodpowiedzi}{ocg62}
				{	\begin{tabular}{p{0.5\textwidth}p{0.3\textwidth}}
				\vspace{0pt}
				\[
					\left[\begin{matrix}4 & 1\\-3 & 1\\-4 & 1\\1 & 1\\3 & 1\end{matrix}\right]
					\cdot \left[\begin{matrix} a \\ b \end{matrix}\right]
					= \left[\begin{matrix}-3\\2\\5\\-1\\-4\end{matrix}\right]
					\quad \Biggm/ \cdot \left(\left(A^T A \right)^{-1} A^T \right)_L
				\]
				\[
					 \left[\begin{matrix} a \\ b \end{matrix}\right] = 
					 \left[\begin{matrix} -1 \\ 0 \end{matrix}\right]
				\]
				&
					\raisebox{-4.9cm}{\resizebox{5.2cm}{!}{\includegraphics{../pics/regresja62}}}
				\end{tabular}
			}

    \end{enumerate}
\newpage % \hspace{1cm}
%--------------------------------------------------------------------------------
    \fancyhead[C]{\textbf{Zestaw nr 10}}
    \begin{enumerate}[label= \textbf{Zadanie \arabic*. }, leftmargin=1cm, align=left, itemsep=0pt]
		\item Rozwiązać równanie w zbiorze liczb zespolonych.\ Sprawdzić rozwiązanie.
			\[
				\left(7 - 4i\right)z = -4 + 3i+\left(7 - 5i\right)z
			\]
			\zOdpowiedziami{\kolorodpowiedzi}{ocg63}
				{$z=3 + 4i.$}

		\item Rozwiązać równanie w zbiorze liczb zespolonych
			\[
				z \left(-2 + 2 i\right) + \left(-3 - i\right) \overline{z} - 1 - 3 i = 0
			\]
			\zOdpowiedziami{\kolorodpowiedzi}{ocg64}
				{$\left(-3 - i\right) \left(x - i y\right) + \left(-2 + 2 i\right) \left(x + i y\right) - 1 - 3 i = 0,$ \\ 
			$\left(-1 - i\right) \left(x \left(2 - 3 i\right) + y \left(1 - 2 i\right) + 2 + i\right) = 0,$\\
			$\left\{
				\begin{array}{c}
					- 5 x - 3 y - 1 = 0\\
					x + y - 3 = 0
				\end{array}
			\right.$ \\
			$z = \left\{ x : -5, \  y : 8\right\}.$}

		\item Rozwiązać równanie w zbiorze liczb zespolonych.\ Sprawdzić jedno z rozwiązań.
			\[
				\left(1 - i\right)z^2 + \left(13 + 7 i\right) z + \left(-6 + 38 i\right)=0
			\]
			\zOdpowiedziami{\kolorodpowiedzi}{ocg65}
				{$\Delta = -8+6i, \quad \sqrt{\Delta}=\pm( 1 + 3 i), \quad z_{1}=-2 - 4 i, \quad z_{2}=-1 - 6 i$}

		\item Dla jakich rzeczywistych wartości parametru $x$ wyznacznik macierzy $A$ jest różny od zera?
			\[
				\textnormal{A=}
				\left[\begin{matrix}-3 & 3 & x + 2\\x + 2 & -3 & 2\\2 x + 1 & -3 & -4\end{matrix}\right]
			\]
			\zOdpowiedziami{\kolorodpowiedzi}{ocg66}
				{$\det A=3 x^{2} + 27 x - 30 \neq 0, \quad x\neq -10, \ x\neq 1, \ $}

		\item Dla jakich rzeczywistych wartości parametru $x$ macierz $A$ posiada odwrotność?
			\[
				\textnormal{A=}\left[\begin{matrix}4 & x + 2\\x + 1 & 2 x + 4\end{matrix}\right]
			\]
		Wyznaczyć macierz odwrotną dla $x=-1.$ Wykonać sprawdzenie.\\
			\zOdpowiedziami{\kolorodpowiedzi}{ocg67}
				{$\det A=- x^{2} + 5 x + 14\neq 0, \quad 
				x\neq -2\ \textnormal{oraz} \ x\neq 7,$\\
				$A(-1)= \left[\begin{matrix}4 & 1\\0 & 2\end{matrix}\right],\ 
				\det A(-1)=8,\ 
				A^{-1}=\frac{1}{8}\left[\begin{matrix}2 & -1\\0 & 4\end{matrix}\right].$}

		\item Rozwiązać równanie:
			\[
				\left[\begin{matrix}1 & -1 & 2 & 2 & 2\\0 & 1 & -1 & -1 & -1\end{matrix}\right]\cdot 
				\left[\begin{matrix}-2 & -1 & 0 & 0 & 0\\1 & 0 & -1 & 1 & -1\end{matrix}\right]^T + 
				2X=
				\left[\begin{matrix}0 & 1\\-1 & 1\\0 & 0\\2 & 0\\2 & 0\end{matrix}\right]^T \cdot
				\left[\begin{matrix}1 & -1\\0 & 0\\0 & -1\\0 & 0\\1 & -1\end{matrix}\right] \cdot X
			\]
			\zOdpowiedziami{\kolorodpowiedzi}{ocg68}
				{$ \left[\begin{matrix}-1 & -1\\-1 & 1\end{matrix}\right] + 
				2X=
				\left[\begin{matrix}2 & -2\\1 & -1\end{matrix}\right] \cdot X, \quad 
				\left[\begin{matrix}-1 & -1\\-1 & 1\end{matrix}\right] = 
				\left[\begin{matrix}0 & -2\\1 & -3\end{matrix}\right] \cdot X $ \\ 
				$X=\frac{1}{2}\left[\begin{matrix}1 & 5\\1 & 1\end{matrix}\right].$}

		\item Wyznaczyć prostą regresji dla punktów 
			\[
				(1,-2),\  (-5,3),\  (-3,4),\  (4,-5),\  (-1,-1).
			\]
			\zOdpowiedziami{\kolorodpowiedzi}{ocg69}
				{	\begin{tabular}{p{0.5\textwidth}p{0.3\textwidth}}
				\vspace{0pt}
				\[
					\left[\begin{matrix}1 & 1\\-5 & 1\\-3 & 1\\4 & 1\\-1 & 1\end{matrix}\right]
					\cdot \left[\begin{matrix} a \\ b \end{matrix}\right]
					= \left[\begin{matrix}-2\\3\\4\\-5\\-1\end{matrix}\right]
					\quad \Biggm/ \cdot \left(\left(A^T A \right)^{-1} A^T \right)_L
				\]
				\[
					 \left[\begin{matrix} a \\ b \end{matrix}\right] = 
					 \left[\begin{matrix} -1 \\ -1 \end{matrix}\right]
				\]
				&
					\raisebox{-4.9cm}{\resizebox{5.2cm}{!}{\includegraphics{../pics/regresja69}}}
				\end{tabular}
			}

    \end{enumerate}
\end{document}
